\documentclass[sigconf]{acmart}

\usepackage{booktabs} % For formal tables

% Editing convenience functions
\newcommand{\matt}[1]{\textcolor{teal}{#1 -m9j} \\}
\graphicspath{{./img/}}

% Copyright
\setcopyright{none}
%\setcopyright{acmcopyright}
%\setcopyright{acmlicensed}
%\setcopyright{rightsretained}
%\setcopyright{usgov}
%\setcopyright{usgovmixed}
%\setcopyright{cagov}
%\setcopyright{cagovmixed}


% DOI
\acmDOI{0}

% ISBN
\acmISBN{0}

%Conference
\acmConference[CSE 544]{UW CSE Databases}{Winter 2018}{Seattle}
\acmYear{2018}
\copyrightyear{2018}

%\acmArticle{4}
\acmPrice{0.00}

% These commands are optional
%\acmBooktitle{Transactions of the ACM Woodstock conference}
%\editor{Dan Suciu}

\begin{document}
\title{Distributed Ledgers for the Cellular Core Network}
%\titlenote{No title note}
%\subtitle{No subtitle}
%\subtitlenote{No subtitle note}


\author{Matt Johnson}
%\authornote{No authornote}
\orcid{0003-2095-0225}

\affiliation{%
  \institution{Paul G. Allen School}
  \city{Seattle}
  \state{Washington}
}
\email{matt9j@cs.uw.edu}

% The default list of authors can be too long for headers.
%\renewcommand{\shortauthors}{B. Trovato et al.}


%\begin{abstract}
%\matt{To be written.}
%\end{abstract}

%
% The code below should be generated by the tool at
% http://dl.acm.org/ccs.cfm
% Please copy and paste the code instead of the example below.
%
\begin{CCSXML}
<ccs2012>
</ccs2012>
\end{CCSXML}

\ccsdesc[100]{Networks~Network reliability}

\keywords{Databases, Community Cellular, CRDT, Blockchain, HSS, LTE, Packet Core}

\maketitle

\section{Project Description}

I would like to explore the challenges and opportunities of combining
concepts from the world of conflict free replicated data types (CRDT)s
with a backing secure distributed ledger. Specifically I would like to
develop a prototype implementation of a distributed cellular network
subscriber server for community cellular networks which will allow
operation when the local rural network is partitioned from the global
internet and network core. Individual nodes would collect and batch
signed state updates locally in a manner consistent with CRDT
principles, and then disseminate them to a backing secure distributed
ledger.

\subsection{Community Cellular Networks} 
Unlike the internet, cellular networks perform robust authentication
and network access control of nodes on the network. This control
allows for cellular networks to provide tighter guarantees on
end-to-end quality of service and track each user's consumption. These
access logs are primarily maintained for the purposes of billing, but
also serve an important role to law enforcement. Community cellular
networks are full cellular networks, using the same technologies and
user devices (cellphones) as national carriers, but scoped for small
rural communities where national carriers are not incentivized to
deploy their own infrastructure.\cite{Heimerllongitudinalstudylocal2015}

Due to their rural nature, community cellular networks often cope with
a slow and intermittent backhaul connection between the edge community
and the network core. Traditional approaches requiring updates to
shared subscriber state do not allow the network to function when
backhaul is down, although in a rural community such local calling
can be valuable or even lifesaving.

\subsection{CRDT}
Conflict-Free Replicated Data Types of a special type of data which
have been shown to provide firm guarantees of Strong Eventual
Consistency.\cite{ShapiroConflictfreereplicateddata2011} New databases exist based on CRDTs designed to
allow hyper-scale web services to respond quickly to queries where
having the exact most up to date data is not essential and the
connection to other database replicas may be unreliable.\cite{DatanetNewCRDT16} These
databases provide a guarantee that strong consistency can be achieved
within a finite time window as soon as connectivity is restored. As
far as I have found, all implementations assume benign nodes.

\subsection{Secure Distributed Ledgers}
Colloquially referred to as ``blockchains,'' secure distributed
ledgers allow for nodes in a network to efficiently share a view of
global state in a completely distributed manner, without a trusted
intermediary, and in a way tolerant to failure and compromise of
individual nodes.\cite{BabuBlockchainTelco2016} In the context of community networking, these
ledgers would facilitate independent communities in a region to band
together into federations with transparent roaming between the
communities.

\section{References \& Related Work}
Many high level whitepapers cover the ``opportunities'' for
blockchains in
telecommunications\cite{BabuBlockchainTelco2016}\cite{BubleyBlockchainTelecomsIndustry2017}\cite{BaeOperatingmiddledigital},
but they seem to all be abstract consulting reports. There is one key
paper\cite{JoverdHSSdistributedPeertoPeer2016a} which explores
building a distributed Home Subscriber Server on a distributed ledger,
but it does not address intermittent backhaul and is only concerned
with user authentication, not user resource tracking as I would like
to attempt. To get a background on the project I would recommend
looking at Jover and Lackey \cite{JoverdHSSdistributedPeertoPeer2016a}
and Shapiro et al.\cite{ShapiroConflictfreereplicateddata2011} as they
are the most relevant to the project.

\section{Tools}

I plan to leverage the open source ledger tools available from the
hyperledger consortium (https://github.com/hyperledger) and my lab's
current community cellular network management
software\cite{Heimerllongitudinalstudylocal2015} and network
testbed. I already have access to these resources.


\section{Project Description}

I would like to explore the challenges and opportunities of combining
concepts from the world of conflict free replicated data types (CRDT)s
with a backing secure distributed ledger. Specifically I would like to
develop a prototype implementation of a distributed cellular network
subscriber server for community cellular networks which will allow
operation when the local rural network is partitioned from the global
internet and network core. Individual nodes would collect and batch
signed state updates locally in a manner consistent with CRDT
principles, and then disseminate them to a backing secure distributed
ledger.

\subsection{Community Cellular Networks}
Unlike the internet, cellular networks perform robust authentication
and network access control of nodes on the network. This control
allows for cellular networks to provide tighter guarantees on
end-to-end quality of service and track each user's consumption. These
access logs are primarily maintained for the purposes of billing, but
also serve an important role to law enforcement. Community cellular
networks are full cellular networks, using the same technologies and
user devices (cellphones) as national carriers, but scoped for small
rural communities where national carriers are not incentivized to
deploy their own infrastructure.\cite{Heimerllongitudinalstudylocal2015}

Due to their rural nature, community cellular networks often cope with
a slow and intermittent backhaul connection between the edge community
and the network core. Traditional approaches requiring updates to
shared subscriber state do not allow the network to function when
backhaul is down, although in a rural community such local calling
can be valuable or even lifesaving.

\subsection{CRDT}
Conflict-Free Replicated Data Types of a special type of data which
have been shown to provide firm guarantees of Strong Eventual
Consistency.\cite{ShapiroConflictfreereplicateddata2011} New databases exist based on CRDTs designed to
allow hyper-scale web services to respond quickly to queries where
having the exact most up to date data is not essential and the
connection to other database replicas may be unreliable.\cite{DatanetNewCRDT16} These
databases provide a guarantee that strong consistency can be achieved
within a finite time window as soon as connectivity is restored. As
far as I have found, all implementations assume benign nodes.

\subsection{Secure Distributed Ledgers}
Colloquially referred to as ``blockchains,'' secure distributed
ledgers allow for nodes in a network to efficiently share a view of
global state in a completely distributed manner, without a trusted
intermediary, and in a way tolerant to failure and compromise of
individual nodes.\cite{BabuBlockchainTelco2016} In the context of community networking, these
ledgers would facilitate independent communities in a region to band
together into federations with transparent roaming between the
communities.

\section{References \& Related Work}
Many high level whitepapers cover the ``opportunities'' for
blockchains in
telecommunications\cite{BabuBlockchainTelco2016}\cite{BubleyBlockchainTelecomsIndustry2017}\cite{BaeOperatingmiddledigital},
but they seem to all be abstract consulting reports. There is one key
paper\cite{JoverdHSSdistributedPeertoPeer2016a} which explores
building a distributed Home Subscriber Server on a distributed ledger,
but it does not address intermittent backhaul and is only concerned
with user authentication, not user resource tracking as I would like
to attempt. To get a background on the project I would recommend
looking at Jover and Lackey \cite{JoverdHSSdistributedPeertoPeer2016a}
and Shapiro et al.\cite{ShapiroConflictfreereplicateddata2011} as they
are the most relevant to the project.

\section{Tools}

I plan to leverage the open source ledger tools available from the
hyperledger consortium (https://github.com/hyperledger) and my lab's
current community cellular network management
software\cite{Heimerllongitudinalstudylocal2015} and network
testbed. I already have access to these resources.

% \begin{acks}

%\end{acks}

\bibliographystyle{ACM-Reference-Format}
\bibliography{blockchain-epc}

\end{document}
